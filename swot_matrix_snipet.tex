\documentclass{company_analysis}

\begin{document}

\companyname{company_logo.png}{Mensch und Maschine}


% %%%%%%%%%%%%%%%%%%%%%%%%%%
% Company Story and Information
% %%%%%%%%%%%%%%%%%%%%%%%%%%
\story
    {The german company \enquote{Mensch und Maschine} resells CAD/E/M
    products by the american company \enquote{AutoDesk} and develops
    own features and Add-ons. The attractive growth in sales, margins
    and dividend needs o continues to be a very good investment. The
    two-pillar business model of a high margin software development and
    lower-margin VAR/project business adds up pretty nicely. Over the
    coming years, M+M must increase it's margins and find ways to reduce
    costs, especially in the VAR-Segment! But the strong balance sheet,
    over 30 years of experience and low exposure to institutional investors
    makes it a great Peter Lynch story.}
\infosection
    {1984}
    {Adi Drotleff}
    {24.594}
    {946}
    {CAD-Software}
    {73,515}
    {---}
    {Fast Grower}
    {Some very good reason}


% %%%%%%%%%%%%%%%%%%%%%%%%%%
% Additional information
% %%%%%%%%%%%%%%%%%%%%%%%%%%

\begin{minipage}{.6\linewidth}
    \faArchive\space\textbf{Products:}
    \begin{enumerate}
        \item Claims to reduce production times
        \item over 100 proucts in CAD/M/E
    \end{enumerate}
    $\Rightarrow$ Needed for building Bridges, Gardens, stadiums, manufacturing
\end{minipage}
% \begin{minipage}{.4\linewidth}
        \begin{piechart}[Revenue 2019]
            {46/Germany, 21/Austria and Switzerland, 25/Europe, 8/Asia and America}
        \end{piechart}
%     \captionof*{figure}{Revenue 2019}
% \end{minipage}



% \begin{minipage}{.4\linewidth}
%     \begin{piechart}{
%             {1/4}/Architecture,
%             {3/4}/Zwei
%     }
%     \end{piechart}
% \end{minipage}



% %%%%%%%%%%%%%%%%%%%%%%%%%%
% SWOT
% %%%%%%%%%%%%%%%%%%%%%%%%%%
\begin{tikzpicture}[
    square/.style={%
        shape=rectangle, minimum width=\tileWidth cm, minimum height=\tileHeight cm,
        inner sep=-1mm, draw
    }, font=\scriptsize\sffamily, thick
]
\draw[black, fill=gray] (-\tileWidth,-\tileHeight) rectangle (\tileWidth,\tileHeight);

    \swot
    { % +++ STRENGTHS +++ %
    \begin{itemize}[leftmargin=*,noitemsep]
        \item Lead by founder Adi Drotleff
        \item Grew profitable with new issuance of shares - Buyback of \num{80592} shares @\euro\num{33,22}
        \item No customer accouts for more than 2\% of revenue
        \item Management owns  over 50\% of the company
        \item Strong balance sheet, ~\euro\num{15}m in debt, \euro\num{32}m CashFlow (9M 2020)
    \end{itemize}
    }
    { % +++ WEAKNESSES +++ %
    \begin{itemize}[leftmargin=*, noitemsep]
        \item Increased outstanding shares: 2014: \num{15439}m to 2019: \num{16820}m shares
        \item Heavily relying on AutoDesk price policy
        \item VAR business expansion into other countries requires new offices $\rightarrow$ costs
        \item Strong Focus in \enquote{D/A/CH} region, maybe too much
        \item Personel risks, specific knowledge is required
    \end{itemize}
    }
    { % +++ OPPORTUNITIES +++ %
    \begin{itemize}[leftmargin=*,noitemsep]
        \item Germany requires BIM-enabled 3D models for public building projects
        \item High-margin internally developed software
        \item Economics of scale in VAR buisiness $\rightarrow$ Higher margins lead to lower costs
    \end{itemize}
    }
    { % +++ THREATS +++ %
    \begin{itemize}[leftmargin=*,noitemsep]
        \item Autodesk main supplier in VAR Segment
        \item Heavily reliant on VAR business (roughly \num{70}\%)
        \item High cost, overhead for VAR business $\rightarrow$ low margins
        \item Takes further capital increases into consideration (For what?!)
    \end{itemize}
    }

    \draw(-0.3,0.3) node {\large\textbf{S}};
    \draw(0.3,0.3) node {\large\textbf{W}};
    \draw(-0.3,-0.3) node {\large\textbf{O}};
    \draw(0.3,-0.3) node {\large\textbf{T}};

\end{tikzpicture}

\newpage
\companyname{company_logo.png}{Mensch und Maschine}



\end{document}