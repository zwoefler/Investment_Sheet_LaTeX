\documentclass{company_analysis}

\begin{document}

\companyname{company_logo.png}{Mensch und Maschine}


% %%%%%%%%%%%%%%%%%%%%%%%%%%
% Company Story and Information
% %%%%%%%%%%%%%%%%%%%%%%%%%%
\story
    {The german company \enquote{Mensch und Maschine} resells CAD/E/M
    products by the american company \enquote{AutoDesk} and develops
    own features and Add-ons. The attractive growth in sales, margins
    and dividend suggest a very stable and sustainably growing business.
    \enquote{MuM} operates a two-pillar business model of a high margin
    software development and lower-margin VAR/project business.
    Over the coming years, \enquote{MuM} must increase it's margins
    and find ways to reduce costs, especially in the VAR-Segment!
    But the strong balance sheet, over 30 years of experience and low
    exposure to institutional investors makes it a great Peter Lynch story.}\hfill
\infosection
    {1984}
    {Adi Drotleff}
    {24.594}
    {946}
    {CAD-Software}
    {73,515}
    {---}
    {Fast Grower}
    {\begin{enumerate}[leftmargin=0.8em, labelsep=0.2em]
        \item Fast Grower
        \item \textgreater 1,6\% Dividend-Yield
        \item Fast growing Dividend 20\% p.a.
    \end{enumerate}\vspace{-0.5cm}
    }

\begin{figure}[h!]
    \subfloat{
        \StickyNote[3cm]{
            \textbf{Goals:}
            \begin{enumerate}
                \item EBITDA margin \textgreater 18\%
                \item 8 - 12\% growth gross profit
                \item Increase Dividend by 18-24 Cents
            \end{enumerate}
        }[6cm][myorange]
        \StickyNote[2cm]{
            How does MuM plan to grow further?!
        }
        [4.5cm]
    }
    \subfloat{
        \piechart[Revenue in \% of 2019]
        {46/Germany, 21/Austria and Switzerland, 25/Europe, 8/Asia and America}
    }
\end{figure}


% %%%%%%%%%%%%%%%%%%%%%%%%%%
% Additional information
% %%%%%%%%%%%%%%%%%%%%%%%%%%

\begin{minipage}{.6\linewidth}
    \faArchive\space\textbf{Products:}
    \begin{enumerate}
        \item Claims to reduce production times
        \item over 100 proucts in CAD/M/E
    \end{enumerate}
    $\Rightarrow$ Needed for building Bridges, Gardens, stadiums, manufacturing

    \smallpiechart[Sales]
    {30/M+M, 70/VAR}
    \smallpiechart[Gross Profit]
    {55/M+M, 45/VAR}
    \smallpiechart[EBITDA]
    {57/M+M, 43/VAR}
\end{minipage}





% %%%%%%%%%%%%%%%%%%%%%%%%%%
% SWOT
% %%%%%%%%%%%%%%%%%%%%%%%%%%
\swot
    { % +++ STRENGTHS +++ %
    \begin{itemize}[leftmargin=*,noitemsep]
        \item Lead by founder Adi Drotleff
        \item Grew profitable with new issuance of shares - Buyback of \num{80592} shares @\euro\num{33,22}
        \item No customer accouts for more than 2\% of revenue
        \item Management owns  over 50\% of the company
        \item Strong balance sheet, ~\euro\num{15}m in debt, \euro\num{32}m CashFlow (9M 2020)
    \end{itemize}
    }
    { % +++ WEAKNESSES +++ %
    \begin{itemize}[leftmargin=*, noitemsep]
        \item Increased outstanding shares: 2014: \num{15439}m to 2019: \num{16820}m shares
        \item Heavily relying on AutoDesk price policy
        \item VAR business expansion into other countries requires new offices $\rightarrow$ costs
        \item Strong Focus in \enquote{D/A/CH} region, maybe too much
        \item Personel risks, specific knowledge is required
    \end{itemize}
    }
    { % +++ OPPORTUNITIES +++ %
    \begin{itemize}[leftmargin=*,noitemsep]
        \item German construction projects require BIM-enabled 3D models
        \item High-margin internally developed software
        \item Economics of scale in VAR buisiness $\rightarrow$ Higher margins lead to lower costs
    \end{itemize}
    }
    { % +++ THREATS +++ %
    \begin{itemize}[leftmargin=*,noitemsep]
        \item Autodesk main supplier in VAR Segment
        \item Heavily reliant on VAR business (roughly \num{70}\%)
        \item High cost, overhead for VAR business $\rightarrow$ low margins
        \item Takes further capital increases into consideration (For what?!)
    \end{itemize}
    }


\newpage
\companyname{company_logo.png}{Mensch und Maschine}

\begin{tabular}{ |p{5cm}||p{3cm}|p{0,7cm}|  }
    \hline
    \multicolumn{3}{|c|}{\raggedleft Q3 2020} \\
    \hline
    Cash   & 19,882    &m€\\
    \hline
    LT Debt& 17,897  & m€\\
    Current LT Debt & 2,195 & m€\\
    Financial Lease Obligations &6,751 & m€\\
    \hline
    Net Debt & -15,141 &m€\\
    \hline
   \end{tabular}

\end{document}